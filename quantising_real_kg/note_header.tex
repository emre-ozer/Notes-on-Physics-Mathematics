\usepackage[utf8]{inputenc}
\usepackage{amsmath, mathrsfs}
\usepackage{amssymb}
\usepackage{enumerate}
\usepackage{tikz}
\usepackage{tensor}
\usepackage[italicdiff]{physics}
\usepackage{amsfonts}
\usepackage{graphicx}
\usepackage{tabularx}
\usepackage[left = 3.5cm, right = 3.5cm, top=3cm, bottom=3cm]{geometry}
\usepackage{hyperref, color}
\hypersetup{
    colorlinks=False,
    linktoc=all,     %set to all if you want both sections and subsections linked
}
\usepackage{mathtools}
\newcommand{\defeq}{\vcentcolon=}
\newcommand{\overbar}[1]{\mkern 1.5mu\overline{\mkern-1.5mu#1\mkern-1.5mu}\mkern 1.5mu}
\newcommand{\infint}{\int_{-\infty}^{\infty}}
\numberwithin{equation}{section}

\usepackage{amsthm}

\theoremstyle{definition}
\newtheorem*{law}{Law}
\newtheorem*{definition}{Definition}
\newtheorem*{proposition}{Proposition}
\newtheorem*{theorem}{Theorem}
\newtheorem*{example}{Example}
\newtheorem*{corollary}{Corollary}
\newtheorem*{lemma}{Lemma}
\newtheorem*{note}{Note}

\usepackage{fancyhdr}
\pagestyle{fancy}


\usetikzlibrary{arrows.meta}
\usetikzlibrary{decorations.markings}
\usetikzlibrary{decorations.pathmorphing}
\usetikzlibrary{positioning}
\usetikzlibrary{fadings}
\usetikzlibrary{intersections}
\usetikzlibrary{cd}
\usetikzlibrary{calc}

\tikzset{>={Latex[length=2mm]}}

\tikzset{
    set arrow inside/.code={\pgfqkeys{/tikz/arrow inside}{#1}},
    set arrow inside={end/.initial=>, opt/.initial=},
    /pgf/decoration/Mark/.style={
        mark/.expanded=at position #1 with
        {
            \noexpand\arrow[\pgfkeysvalueof{/tikz/arrow inside/opt}]{\pgfkeysvalueof{/tikz/arrow inside/end}}
        }
    },
    arrow inside/.style 2 args={
        set arrow inside={#1},
        postaction={
            decorate,decoration={
                markings,Mark/.list={#2}
            }
        }
    },
}
\def\centerarc[#1](#2)(#3:#4:#5)% Syntax: [draw options] (center) (initial angle:final angle:radius)
    { \draw[#1] ($(#2)+({#5*cos(#3)},{#5*sin(#3)})$) arc (#3:#4:#5); }