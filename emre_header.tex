\newcommand{\R}{\mathbb{R}}
\newcommand{\C}{\mathbb{C}}
\newcommand{\Z}{\mathbb{Z}}
\newcommand{\N}{\mathbb{N}}
%\renewcommand{\dv}[3][]{\frac{d^{#1} #2}{d#3^{#1}}}
\newcommand{\dotp}[2]{#1 \cdot #2}
\renewcommand{\cp}[2]{#1 \times #2}
\renewcommand{\grad}[1]{\vec{\nabla} #1}
\renewcommand{\div}[1]{\vec{\nabla} \cdot #1}
\renewcommand{\curl}[1]{\vec{\nabla} \times #1}
\renewcommand{\infint}{\int_{-\infty}^{\infty}}
\newcommand{\oinfint}{\int_0^\infty}
\renewcommand{\implies}{\quad \Rightarrow \quad}
\newcommand{\qqr}[1]{\quad \text{#1}}
\newcommand{\equivalent}{\quad \Leftrightarrow \quad}

% tikz commands
\usetikzlibrary{shapes.misc}
\usetikzlibrary{math}
\tikzset{cross/.style={cross out, draw=black, fill=none,minimum size=2*(#1-\pgflinewidth), inner sep=0pt, outer sep=0pt}, cross/.default={2pt}}
\def\vectorin[#1](#2)% Syntax: [size in pt] (coordinate)
{ \filldraw[fill=white] (#2) circle (sqrt 2*#1) node[cross= #1] {}; }
%{ \draw (#2) circle (sqrt 2*#1) node[cross= #1, fill=white] {}; }
\def\vectorout[#1](#2)
{ \filldraw[fill=white] (#2) circle(5*#1 pt) node[circle,fill=black,inner sep=#1pt] {};}