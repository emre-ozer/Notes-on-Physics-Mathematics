\documentclass{article}
\usepackage[utf8]{inputenc}
\usepackage{amsmath, mathrsfs}
\usepackage{amssymb}
\usepackage{enumerate}
\usepackage{tikz}
\usepackage{tensor}
\usepackage[italicdiff]{physics}
\usepackage{amsfonts}
\usepackage{graphicx}
\usepackage{tabularx}
\usepackage[left = 3.5cm, right = 3.5cm, top=3cm, bottom=3cm]{geometry}
\usepackage{hyperref, color}
\hypersetup{
    colorlinks=False,
    linktoc=all,     %set to all if you want both sections and subsections linked
}
\usepackage{mathtools}
\newcommand{\defeq}{\vcentcolon=}
\newcommand{\overbar}[1]{\mkern 1.5mu\overline{\mkern-1.5mu#1\mkern-1.5mu}\mkern 1.5mu}
\newcommand{\infint}{\int_{-\infty}^{\infty}}
\numberwithin{equation}{section}

\usepackage{amsthm}

\theoremstyle{definition}
\newtheorem*{law}{Law}
\newtheorem*{definition}{Definition}
\newtheorem*{proposition}{Proposition}
\newtheorem*{theorem}{Theorem}
\newtheorem*{example}{Example}
\newtheorem*{corollary}{Corollary}
\newtheorem*{lemma}{Lemma}
\newtheorem*{note}{Note}

\usepackage{fancyhdr}
\pagestyle{fancy}


\usetikzlibrary{arrows.meta}
\usetikzlibrary{decorations.markings}
\usetikzlibrary{decorations.pathmorphing}
\usetikzlibrary{positioning}
\usetikzlibrary{fadings}
\usetikzlibrary{intersections}
\usetikzlibrary{cd}
\usetikzlibrary{calc}

\tikzset{>={Latex[length=2mm]}}

\tikzset{
    set arrow inside/.code={\pgfqkeys{/tikz/arrow inside}{#1}},
    set arrow inside={end/.initial=>, opt/.initial=},
    /pgf/decoration/Mark/.style={
        mark/.expanded=at position #1 with
        {
            \noexpand\arrow[\pgfkeysvalueof{/tikz/arrow inside/opt}]{\pgfkeysvalueof{/tikz/arrow inside/end}}
        }
    },
    arrow inside/.style 2 args={
        set arrow inside={#1},
        postaction={
            decorate,decoration={
                markings,Mark/.list={#2}
            }
        }
    },
}
\def\centerarc[#1](#2)(#3:#4:#5)% Syntax: [draw options] (center) (initial angle:final angle:radius)
    { \draw[#1] ($(#2)+({#5*cos(#3)},{#5*sin(#3)})$) arc (#3:#4:#5); }
\newcommand{\R}{\mathbb{R}}
\newcommand{\C}{\mathbb{C}}
\newcommand{\Z}{\mathbb{Z}}
\newcommand{\N}{\mathbb{N}}
%\renewcommand{\dv}[3][]{\frac{d^{#1} #2}{d#3^{#1}}}
\newcommand{\dotp}[2]{#1 \cdot #2}
\renewcommand{\cp}[2]{#1 \times #2}
\renewcommand{\grad}[1]{\vec{\nabla} #1}
\renewcommand{\div}[1]{\vec{\nabla} \cdot #1}
\renewcommand{\curl}[1]{\vec{\nabla} \times #1}
\renewcommand{\infint}{\int_{-\infty}^{\infty}}
\newcommand{\oinfint}{\int_0^\infty}
\renewcommand{\implies}{\quad \Rightarrow \quad}
\newcommand{\qqr}[1]{\quad \text{#1}}
\newcommand{\equivalent}{\quad \Leftrightarrow \quad}

% tikz commands
\usetikzlibrary{shapes.misc}
\usetikzlibrary{math}
\tikzset{cross/.style={cross out, draw=black, fill=none,minimum size=2*(#1-\pgflinewidth), inner sep=0pt, outer sep=0pt}, cross/.default={2pt}}
\def\vectorin[#1](#2)% Syntax: [size in pt] (coordinate)
{ \filldraw[fill=white] (#2) circle (sqrt 2*#1) node[cross= #1] {}; }
%{ \draw (#2) circle (sqrt 2*#1) node[cross= #1, fill=white] {}; }
\def\vectorout[#1](#2)
{ \filldraw[fill=white] (#2) circle(5*#1 pt) node[circle,fill=black,inner sep=#1pt] {};}

\DeclareMathOperator{\sgn}{sgn}

\title{Variation of Matrix Determinant}
\date{}
\author{Emre Özer}
\begin{document}
	\maketitle
	\tableofcontents
\section{Maths: Derivation}
Let $ M $ be an $ n \times n $ matrix with components $ M_{ij} $ and inverse $ M^{-1} $ with components $ M^{ij} $, such that
\begin{equation} \label{eq:1}
	\sum_{j} M^{-1}_{ij}M_{jk} = \delta_{ik}.
\end{equation}
We are interested in the variation $ \delta \det M $. In terms of $ \delta M_{ij} $, this is
\begin{equation}
	\delta \det M = \sum_{ij} \pdv{\det M}{M_{ij}} \delta M_{ij}.
\end{equation}
One expression for $ \det M $ is
\begin{equation}
	\det M  = \sum_{\pi \in S_{n}} \sgn(\pi) \prod_{k} M_{k\pi(k)},
\end{equation}
where $ \pi $ are permutations and $ S_{n} $ is the symmetric group of degree $ n $. Then,
\begin{equation}
	\pdv{\det M}{M_{ij}} = \sum_{\pi \in S_{n}} \sgn(\pi) \pdv{M_{ij}}(\prod_{k} M_{k\pi(k)}) = \sum_{\pi \in S_{n}} \sgn(\pi) \delta_{j\pi(i)} \prod_{k\neq i} M_{k\pi(k)}.
\end{equation}
Now, use equation \eqref{eq:1} to substitute for $ \delta_{j \pi(i)} $:
\begin{equation}
	\pdv{\det M}{M_{ij}} = \sum_{\pi \in S_{n}} \sgn(\pi) \qty(\sum_{\ell} M^{-1}_{j \ell} M_{\ell \pi(i)} )  \prod_{k\neq i} M_{k\pi(k)} = \sum_{\ell} M^{-1}_{j\ell} \sum_{\pi \in S_{n}} \sgn(\pi) \prod_{k\neq i} M_{k\pi(k)} M_{\ell \pi(i)}.
\end{equation}
The trick here is to separate the sum over $ \ell $ into two parts: $ \ell = i $ and $ \ell \neq i $. The $ \ell = i $ part gives
\begin{equation}
	M^{-1}_{ji} \sum_{\pi \in S_{n}} \sgn(\pi) \prod_{k\neq i} M_{k \pi(k)} M_{i \pi(i)} = 	M^{-1}_{ji} \sum_{\pi \in S_{n}} \sgn(\pi) \prod_{k} M_{k \pi(k)} = M^{-1}_{ji} \det M.
\end{equation}
The $ \ell \neq i $ part is
\begin{equation} \label{eq:7}
	\sum_{\ell \neq i} M^{-1}_{j\ell} \sum_{\pi \in S_{n}} \sgn(\pi) \prod_{k\neq i} M_{k\pi(k)} M_{\ell \pi(i)} = \sum_{\ell \neq i} M^{-1}_{j\ell} \sum_{\pi \in S_{n}} \sgn(\pi) \prod_{k\neq \ell, i} M_{k\pi(k)} M_{\ell \pi(\ell)} M_{\ell \pi(i)},
\end{equation}
where we separated the $ \ell^{\text{th}}$ term from product. Now, here is the key point: the expression is \textit{symmetric} under the exchange $ \pi(i) \leftrightarrow \pi(\ell) $ due to the $ M_{\ell \pi(\ell)}M_{\ell \pi(i)} $ term. 
\par
For any $ \pi \in S_{n} $ with $ \sgn(\pi) = +1 $, we can construct a unique $ \pi' \in S_{n} $ with $ \sgn(\pi') = -1 $ by setting
\begin{equation}
	\pi'(j) = \begin{cases}
	\pi(\ell) & j = i, \\
	\pi(i) & j = \ell, \\
	\pi(j) & \text{otherwise}.
	\end{cases}
\end{equation}
Moreover, the set of all $ \pi' $ is the set of all odd permutations of degree $ n $. Hence, equation \eqref{eq:7} is identically zero. This gives the result
\begin{equation}
	\pdv{\det M}{M_{ij}} = M^{-1}_{ji} \det M \quad\Rightarrow\quad \delta \det M = \det M \sum_{ij} M^{-1}_{ji} \delta M_{ij}.
\end{equation}
\section{Applications to GR}
\subsection{Covariant divergence}
Consider the metric $ g_{\mu \nu} $ with inverse $ g^{\mu \nu} $. Let $ g = \abs{\det g_{\mu \nu}} $. Then, we have
\begin{equation}
	\delta g = g g^{\mu \nu} \delta g_{\mu \nu}
\end{equation}
In particular, this implies
\begin{equation}
	\partial_{\lambda} g = \pdv{g}{g_{\mu \nu}} \partial_{\lambda} g_{\mu \nu} = g g^{\mu \nu} \partial_{\lambda} g_{\mu \nu}.
\end{equation}
The Levi-Civita connection $ \Gamma\indices{^{\mu}_{\nu \lambda}} $ obeys
\begin{equation}
	\partial_{\lambda} g_{\mu \nu} = \Gamma_{\mu \nu \lambda} + \Gamma_{\nu \mu \lambda}.
\end{equation}
Using these, we can obtain a useful expression for the covariant divergence of a vector field
\begin{equation}
	\nabla_{\mu} V^{\mu} = \partial_{\mu} V^{\mu} + \Gamma\indices{^{\mu}_{\mu \lambda}} V^{\lambda}.
\end{equation}
Taking a closer look at the connection:
\begin{equation}
	\Gamma\indices{^{\mu}_{\mu \lambda}} =  g^{\mu \nu} \Gamma_{\nu \mu \lambda} = \frac{1}{2} g^{\mu \nu} \partial_{\lambda} g_{\mu\nu} = \frac{1}{\sqrt{g}}\partial_{\lambda} \sqrt{g}.
\end{equation}
Substituting this to the expression above yields the result:
\begin{equation} \label{eq:covdiv}
	\nabla_{\mu} V^{\mu} = \frac{1}{\sqrt{g}} \partial_{\mu} \qty(\sqrt{g}V^{\mu}).
\end{equation}
\subsection{Covariant Laplacian of scalar}
The Laplacian of a scalar is defined covariantly as
\begin{equation}
	\Box \phi = \nabla^{\mu} \nabla_{\mu} \phi = g^{\mu \nu} \nabla_{\mu} \nabla_{\mu} \phi.
\end{equation}
Noting that the metric is covariantly constant, i.e. $ \nabla_{\lambda} g_{\mu\nu} = 0 $, equation \eqref{eq:covdiv} implies
\begin{equation} \label{covlap}
	g^{\mu \nu} \nabla_{\mu} \nabla_{\mu} \phi = \nabla_{\mu} g^{\mu\nu} \partial_{\nu} \phi = \frac{1}{\sqrt{g}} \partial_{\mu} \qty(\sqrt{g} g^{\mu\nu} \partial_{\nu} \phi).
\end{equation}
\subsection{Minimal coupling: Klein-Gordon}
The special relativistic Klein-Gordon action is
\begin{equation}
	\mathcal{S}[\phi] = \int \dd[4]{x} \qty[-\tfrac{1}{2}\eta^{\mu\nu} \partial_{\mu}\phi \partial_{\nu} \phi - \tfrac{1}{2} m^{2} \phi^{2}],
\end{equation}
with metric convention $ \eta = \text{diag}(-1,+1,+1,+1) $. The resulting equation of motion is
\begin{equation}
	(\Box_{\eta} - m^{2}) \phi = 0,
\end{equation}
where $ \Box_{\eta} = \eta^{\mu\nu}\partial_{\mu}\partial_{\nu} $. Now, we write generally covariant action:
\begin{equation}
	\mathcal{S}[\phi, g_{\mu\nu}] = \int \dd[4]{x} \sqrt{g} \qty[-\tfrac{1}{2} g^{\mu\nu}\partial_{\mu}\phi \partial_{\nu}\phi - \tfrac{1}{2} m^{2} \phi^{2}]
\end{equation}
Varying this action with respect to $ \phi $ yields
\begin{align}
	\begin{split}
		\delta \mathcal{S} &= -\int \dd[4]{x} \sqrt{g} \qty[ g^{\mu\nu} \partial_{\nu} \phi \partial_{\mu} \delta \phi + m^{2} \phi \delta \phi] \\
		&= \int \dd[4]{x} \sqrt{g} \qty[ \frac{1}{\sqrt{g}} \partial_{\mu} \qty( \sqrt{g} g^{\mu \nu} \partial_{\nu} \phi ) - m^{2} \phi ] \delta \phi + \text{surface term} \\
		&= \int \dd[4]{x} \sqrt{g} \qty[\Box_{g} \phi - m^{2} \phi] \delta \phi = 0,
	\end{split}
\end{align}
where we used equation \eqref{covlap} going from line 2 to 3. Hence, the covariant equation of motion is
\begin{equation}
	\qty(\Box_{g} - m^{2}) \phi = 0.
\end{equation}

\subsection{Minimal coupling: Maxwell}
Vacuum special relativistic Maxwell action is
\begin{equation}
	\mathcal{S}[A_{\nu}] = -\frac{1}{4} \int \dd[4]{x}  F^{\mu \nu} F_{\mu \nu} = -\frac{1}{4} \int \dd[4]{x}  \eta^{\mu \alpha} \eta^{\nu \beta} F_{\alpha \beta} F_{\mu \nu},
\end{equation}
with field tensor $ F_{\mu \nu} = \partial_{\mu}A_{\nu} - \partial_{\nu} A_{\mu} $. The correct way to proceed is to keep the field tensor definition fixed, i.e. don't replace $ \partial \to \nabla $. Instead, simply consider the action
\begin{equation}
	\mathcal{S}[A_{\nu}, g_{\mu \nu}] = -\frac{1}{4} \int \sqrt{g}\dd[4]{x}  g^{\mu \alpha} g^{\nu \beta} F_{\alpha \beta} F_{\mu \nu}. 
\end{equation}
Varying the action with respect to $ A_{\nu} $:
\begin{align}
	\begin{split}
		\delta \mathcal{S} &= -\frac{1}{2} \int \sqrt{g}\dd[4]{x}  g^{\mu \alpha} g^{\nu \beta} F_{\alpha \beta} \delta F_{\mu \nu} \\
		&= -\frac{1}{2} \int \sqrt{g}\dd[4]{x}  g^{\mu \alpha} g^{\nu \beta} F_{\alpha \beta} (\partial_{\mu} \delta A_{\nu} - \partial_{\nu} \delta A_{\mu}) \\
		&= -\int \sqrt{g}\dd[4]{x}  g^{\mu \alpha} g^{\nu \beta} F_{\alpha \beta} \partial_{\mu} \delta A_{\nu} \\
		&= \int \dd[4]{x} \partial_{\mu} \qty(\sqrt{g} g^{\mu \alpha} g^{\nu \beta} F_{\alpha \beta}) \delta A_{\nu} + \text{surface term} \\
		&= .\int \dd[4]{x} \partial_{\mu} \qty(\sqrt{g} F^{\mu \nu}) \delta A_{\nu} = 0.
	\end{split}
\end{align}
The resulting equation of motion is
\begin{equation}
	\partial_{\mu} \qty(\sqrt{g} F^{\mu \nu}) = 0.
\end{equation}
At first glance this doesn't look equivalent to $ \nabla_{\mu} F^{\mu \nu} = 0 $, which is what we may have expected. It turns out that they are equal due to antisymmetry of $ F_{\mu \nu} $:
\begin{equation}
	\nabla_{\mu} F^{\mu \nu} = \partial_{\mu} F^{\mu \nu} + \Gamma\indices{^{\mu}_{\mu \lambda}} F^{\lambda \nu} + \Gamma\indices{^{\nu}_{\mu \lambda}} F^{\mu \lambda}.
\end{equation}
The last term vanishes since $ \Gamma\indices{^{\nu}_{\mu \lambda}} $ is symmetric in $ \mu \lambda $, and $ F^{\mu \lambda} $ antisymmetric. Hence, we have
\begin{equation}
	\nabla_{\mu} F^{\mu \nu} = \frac{1}{\sqrt{g}} \partial_{\mu} \qty(\sqrt{g} F^{\mu \nu}) = 0 \quad \Leftrightarrow \quad \partial_{\mu} \qty(\sqrt{g} F^{\mu \nu}) = 0.
\end{equation}
\begin{note}
	Using Euler-Lagrange equation to obtain equations of motion instead of varying the action directly is a waste of time and effort. If you don't believe me, have a go at the above derivation.
\end{note}


\end{document}


















